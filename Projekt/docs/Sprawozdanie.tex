\documentclass[11pt]{article}
\usepackage{graphicx} % Required for inserting images
\usepackage{float}
\usepackage{adjustbox}
\usepackage{subfig}
\usepackage{booktabs}
\usepackage{siunitx}
\usepackage[a4paper,top=2cm,bottom=2cm,left=3cm,right=3cm,marginparwidth=1.75cm]{geometry}
\usepackage{tikz}
\usepackage{pgfplots}
\usepackage{pgfplotstable}
\usepackage{amsmath}
\usepackage{tabularx}
\usepackage{array}
\newcolumntype{Y}{>{\centering\arraybackslash}X}
\usepackage{xcolor,colortbl}
\usepackage{titling}
\usepackage{url}


\pgfplotsset{compat=newest}
\usepgfplotslibrary{external}
\tikzexternalize[prefix=tikz/]

\title{ELA2 - Projekt}
\author{Piotr Pokornowski 325061}
\date{\today}


\begin{document}

\begin{titlingpage}
    \maketitle
\end{titlingpage}

\section{Sekcja cyfrowa}
\subsection{Opis układu}
Zaprojektowana przetwornica obniża napięcie z \SI{10}{\V} do \SI{5}{\V} i działa dla prądu maksymalnego \SI{5}{\A}. Jej zadaniem jest zasilanie cyfrowej sekcji układu, z tego powodu minimalizacja tętnień była celem drugorzędnym, a priorytetem stało się uzyskanie jak największej sprawności \textemdash \ dla maksymalnego obciążenia prądem \SI{5}{\A} udało się uzyskać sprawność na poziomie $\sim \SI{94.3}{\percent}$.

\begin{figure}[H]
    \centering
    \captionsetup{justification=centering}
    \begin{adjustbox}{width=1.1\textwidth,center}
        \subfloat[Schemat układu]{
            \includegraphics[width=\textwidth]{./figures/digital/digital.png}
            \label{fig:subfig1}
        }
        \qquad
        \subfloat[Przebiegi czasowe prądu i napięcia na wyjściu układu \\ przy maksymalnym obciążeniu]{
            \begin{tikzpicture}
                \pgfplotsset{
                    scale only axis,
                    scaled x ticks=base 10:3,
                    xmin=0, xmax=0.008
                }

                \begin{axis}[
                        grid=both,
                        axis y line*=left,
                        ymin=-1, ymax=6,
                        xlabel={Czas $t$ [s]},
                        ylabel={Napięcie wyjściowe $V_{\text{OUT}}$ [V]},
                    ]
                    \addplot[green, thick] table {./figures/digital/vout/segment_1.csv};\label{vout}
                    \addplot[green, thick] table {./figures/digital/vout/segment_2.csv};
                    \label{plot:vout}
                \end{axis}

                \begin{axis}[
                        axis y line*=right,
                        axis x line=none,
                        ymin=-1,
                        ymax=6,
                        ytick pos=right,
                        yticklabel pos=right,
                        ytick distance=1,
                        ylabel={Prąd wyjściowy $I_{\text{OUT}}$ [A]},
                        legend style={at={(0.95,0.05)},anchor=south east},
                    ]
                    \addlegendimage{/pgfplots/refstyle=vout}
                    \addlegendentry{$V_{OUT}$}
                    \addplot[red, thick] table {./figures/digital/iout/segment_1.csv};\label{iout}
                    \addplot[red, thick] table {./figures/digital/iout/segment_2.csv};

                    \addlegendimage{/pgfplots/refstyle=iout}
                    \addlegendentry{$I_{OUT}$}
                    \label{plot:current}
                \end{axis}
            \end{tikzpicture}

            \label{fig:subfig2}
        }
    \end{adjustbox}
\end{figure}

\subsection{Wybór elementów}
\subsubsection{Wybór LT8640}
Wykorzystany układ LT8640 został wybrany, ponieważ spełniał wymagania projektowe oraz był dostępny w dużych ilościach wśród dostawców. Dodatkowo, Analog Devices sugeruje go jako jeden z układów do wykorzystania przy nowych projektach.

\subsubsection{Dobór cewki}
Cewka została dobrana zgodnie ze wzorem
\[
    L = \frac{V_{OUT} + V_{SW(BOT)}}{f_{SW}} \approx \SI{3.6}{\micro\henry}
\]
dostępnym w nocie katalogowej. W układzie znalazła się więc najbliższa z szeregu cewka o wartości \SI{3.3}{\micro\henry}. Podobna cewka o takiej samej wartości jest używana w przykładowym układzie producenta.

\subsubsection{Dobór kondensatorów}
Kondensatory zostały dobrane zgodnie z zaleceniami producenta układu, jednocześnie biorąc pod uwagę spadek pojemności. Wszystkie użyte kondensatory to kondensatory MLCC, z wyjątkiem dwóch dużych kondensatorów wyjściowych. Zamiast MLCC użyte są tam aluminiowe kondensatory elektrolityczne, co pozwoliło na znaczną redukcję kosztów przy zachowaniu dobrych parametrów. Wykorzystanie w tym miejscu trudno dostępnych kondensatorów ceramicznych o pojemności \SI{47}{\micro\farad} byłoby nieopłacalne, gdyż kosztowały one więcej niż cała reszta elementów w układzie.

\subsection{Wyniki symulacji}
\begin{table}[H]
    \centering
    \caption{Porównanie wyników symulacji w zależności od obciążenia}
    \label{tab:efficiency_ripple}
    \begin{tabular}{ccccc}
        \toprule
        \textbf{Prąd obciążenia [mA]} & \textbf{Sprawność [\%]} & $\mathbf{\Delta{V_{IN}} \ [mV]}$ & $\mathbf{\Delta{V_{OUT}} \ [mV]}$ & $\mathbf{V_{AVG} \ [V]}$ \\
        \midrule
        0.0                           & 0.00                    & 0.00                             & 0.68                              & 5.03                     \\
        0.5                           & 97.76                   & 147.75                           & 69.52                             & 4.96                     \\
        1.0                           & 97.99                   & 234.75                           & 70.82                             & 4.96                     \\
        1.5                           & 97.70                   & 316.64                           & 72.18                             & 4.96                     \\
        2.0                           & 97.28                   & 402.93                           & 70.74                             & 4.95                     \\
        2.5                           & 96.82                   & 485.24                           & 71.01                             & 4.95                     \\
        3.0                           & 96.33                   & 565.54                           & 70.95                             & 4.95                     \\
        3.5                           & 95.92                   & 649.69                           & 71.78                             & 4.95                     \\
        4.0                           & 95.31                   & 730.42                           & 73.02                             & 4.95                     \\
        4.5                           & 94.80                   & 813.73                           & 74.20                             & 4.95                     \\
        5.0                           & 94.28                   & 899.40                           & 75.92                             & 4.95                     \\
        \bottomrule
    \end{tabular}
\end{table}

\subsection{Wnioski}
Wykorzystanie przetwornicy impulsowej pozwoliło na uzyskanie bardzo dużej sprawności, co byłoby niemożliwe w przypadku stabilizatora liniowego. Wadą takiego rozwiązania jest jednak większe tętnienie napięcia wyjściowego, co w przypadku zasilania cyfrowego układu nie jest problemem. Kolejną zaletą jest również stabilność napięcia wyjściowego przy zmianach obciążenia, co byłoby niemożliwe gdy do obniżenia napięcia wykorzystany byłby np. prosty dzielnik napięciowy.

% ----------------------------------------------

\section{Sekcja analogowa I}
\subsection{Opis układu}
Zaprojektowany układ obniża napięcie z \SI{10}{\V} do \SI{5}{\V} i działa dla prądu maksymalnego \SI{250}{\milli\A}. Jego zadaniem jest zasilanie analogowej sekcji układu, która powinna mieć jak najmniejsze tętnienia. Z tego powodu ten układ został wykonany z użyciem wyłącznie stabilizatora, co powinno zapewnić (przynajmniej w teorii) zerowe tętnienia.

\begin{figure}[H]
    \centering
    \captionsetup{justification=centering}
    \begin{adjustbox}{width=1.1\textwidth,center}
        \subfloat[Schemat układu]{
            \includegraphics[width=\textwidth]{./figures/analog_1/schematic.png}
            \label{fig:subfig1}
        }
        \qquad
        \subfloat[Przebiegi czasowe prądu i napięcia na wyjściu układu \\ przy maksymalnym obciążeniu]{
            \begin{tikzpicture}
                \pgfplotsset{
                    scale only axis,
                    scaled x ticks=base 10:3,
                    xmin=0, xmax=0.008
                }

                \begin{axis}[
                        grid=both,
                        axis y line*=left,
                        ymin=-1, ymax=6,
                        xlabel={Czas $t$ [s]},
                        ylabel={Napięcie wyjściowe $V_{\text{OUT}}$ [V]},
                    ]
                    \addplot[green, thick] table {./figures/analog_1/vout/vout.txt};\label{vout}
                    \label{plot:vout}
                \end{axis}

                \begin{axis}[
                        axis y line*=right,
                        axis x line=none,
                        ymin=-0.050,
                        ymax=.300,
                        ytick pos=right,
                        yticklabel pos=right,
                        ytick = {-0.05, 0, 0.05, 0.1, 0.15, 0.2, 0.25, 0.3},
                        yticklabels = {-50, 0, 50, 100, 150, 200, 250, 300},
                        ylabel={Prąd wyjściowy $I_{\text{OUT}}$ [mA]},
                        legend style={at={(0.95,0.05)},anchor=south east},
                    ]
                    \addlegendimage{/pgfplots/refstyle=vout}
                    \addlegendentry{$V_{OUT}$}
                    \addplot[red, thick] table {./figures/analog_1/iout/iout.txt};\label{iout}

                    \addlegendimage{/pgfplots/refstyle=iout}
                    \addlegendentry{$I_{OUT}$}
                    \label{plot:current}
                \end{axis}
            \end{tikzpicture}
            \label{fig:subfig2}
        }
    \end{adjustbox}
\end{figure}

\subsection{Wybór elementów}
\subsubsection{Wybór stabilizatora}
W układzie został wykorzystany stabilizator LT3012, ponieważ jego parametry spełniają wymagania projektowe, a producent poleca go do wykorzystania w nowych projektach.

\subsubsection{Dobór kondensatorów}
Kondensatory zostały dobrane zgodnie z zaleceniami producenta układu, jednocześnie biorąc pod uwagę spadek pojemności. Wszystkie użyte kondensatory to kondensatory MLCC. Dodatkową kwestią przy ich wyborze, był fakt, że są one już wykorzystywane w sekcji cyfrowej, co pozwoliło na zmniejszenie kosztów.
\subsection{Wyniki symulacji}
\begin{table}[H]
    \centering
    \caption{Porównanie wyników symulacji w zależności od obciążenia}
    \label{tab:efficiency_ripple}
    \begin{tabular}{ccccc}
        \toprule
        \textbf{Prąd obciążenia [mA]} & \textbf{Sprawność [\%]} & $\mathbf{\Delta{V_{IN}} \ [mV]}$ & $\mathbf{\Delta{V_{OUT}} \ [mV]}$ & $\mathbf{V_{AVG} \ [V]}$ \\
        \midrule
        0.0                           & 0                       & 0.0                              & 2.82                              & 5.07                     \\
        25.0                          & 48.38                   & 0.0                              & 0.14                              & 5.01                     \\
        50.0                          & 48.45                   & 0.0                              & 0.14                              & 5.01                     \\
        75.0                          & 48.48                   & 0.0                              & 0.14                              & 5.01                     \\
        100.0                         & 48.49                   & 0.0                              & 0.14                              & 5.01                     \\
        125.0                         & 48.50                   & 0.0                              & 0.14                              & 5.01                     \\
        150.0                         & 48.50                   & 0.0                              & 0.14                              & 5.01                     \\
        175.0                         & 48.51                   & 0.0                              & 0.14                              & 5.01                     \\
        200.0                         & 48.51                   & 0.0                              & 0.14                              & 5.00                     \\
        225.0                         & 48.52                   & 0.0                              & 0.14                              & 5.00                     \\
        250.0                         & 48.52                   & 0.0                              & 0.14                              & 5.00                     \\
        \bottomrule
    \end{tabular}
\end{table}

\subsection{Wnioski}
Układ wykonany z użyciem samego stabilizatora charakteryzuje się pomijalnymi tętnieniami napięcia wyjściowego. Wadą takiego rozwiązania jest jednak mniejsza sprawność w porównaniu do przetwornicy impulsowej. W przypadku zasilania analogowej sekcji układu, gdzie priorytetem jest stabilność napięcia, a nie sprawność, jest to jednak akceptowalne rozwiązanie.
% ----------------------------------------------

\section{Sekcja analogowa II}
\subsection{Opis układu}
Zaprojektowana przetwornica podnosi napięcie z \SI{10}{\V} do \SI{12}{\V} i działa dla prądu maksymalnego \SI{100}{\milli\A}. Jej zadaniem jest zasilanie analogowej sekcji układu, która powinna mieć jak najmniejsze tętnienia. Układ wykorzystuje przetwornicę LT8362.

\begin{figure}[H]
    \centering
    \captionsetup{justification=centering}
    \begin{adjustbox}{width=1.2\textwidth,center}
        \subfloat[Schemat układu]{
            \includegraphics[width=\textwidth]{./figures/analog_2/schematic.png}
            \label{fig:subfig1}
        }
        \qquad
        \subfloat[Przebiegi czasowe prądu i napięcia na wyjściu układu \\ przy maksymalnym obciążeniu]{
            \begin{tikzpicture}
                \pgfplotsset{
                    scale only axis,
                    scaled x ticks=base 10:3,
                    xmin=0, xmax=0.005
                }

                \begin{axis}[
                        grid=both,
                        axis y line*=left,
                        ymin=-1, ymax=13,
                        xlabel={Czas $t$ [s]},
                        ylabel={Napięcie wyjściowe $V_{\text{OUT}}$ [V]},
                    ]
                    \addplot[green, thick] table {./figures/analog_2/vout/segment_1.csv};\label{vout}
                    \addplot[green, thick] table {./figures/analog_2/vout/segment_2.csv};
                    \label{plot:vout}
                \end{axis}

                \begin{axis}[
                        axis y line*=right,
                        axis x line=none,
                        ymin=-0.01,
                        ymax=0.13,
                        ytick pos=right,
                        yticklabel pos=right,
                        ytick = {0, 0.02, 0.04, 0.06, 0.08, 0.1, 0.12},
                        yticklabels = {0, 20, 40, 60, 80, 100, 120},
                        ylabel={Prąd wyjściowy $I_{\text{OUT}}$ [mA]},
                        legend style={at={(0.95,0.05)},anchor=south east},
                    ]
                    \addlegendimage{/pgfplots/refstyle=vout}
                    \addlegendentry{$V_{OUT}$}
                    \addplot[red, thick] table {./figures/analog_2/iout/segment_1.csv};\label{iout}
                    \addplot[red, thick] table {./figures/analog_2/iout/segment_2.csv};

                    \addlegendimage{/pgfplots/refstyle=iout}
                    \addlegendentry{$I_{OUT}$}
                    \label{plot:current}
                \end{axis}
            \end{tikzpicture}

            \label{fig:subfig2}
        }
    \end{adjustbox}
\end{figure}

\subsection{Wybór elementów}
\subsubsection{Wybór LT8362}
Przetwornica obniżająca została wykonana w topologii SEPIC, ponieważ charakteryzują sie one mniejszymi tętnieniami napięcia wyjściowego niż klasyczne przetwornice typu boost. Układ LT8362 został wybrany, ponieważ spełniał wymagania projektowe oraz był dostępny w dużych ilościach wśród dostawców. Dodatkowo, Analog Devices sugeruje go jako jeden z układów do wykorzystania przy nowych projektach.

\subsubsection{Dobór cewki}
Cewka o wartości \SI{4.7}{\micro\henry} została dobrana zgodnie ze wzorami dostępnym w nocie katalogowej, jednocześnie sugerując się przykładowym układem producenta, który użył takiej samej cewki w układzie o $V_{OUT} = \SI{12}{\V}$.

\subsection{Wybór diody}
W układzie wykorzystana została dioda polecana przez producenta układu.

\subsubsection{Dobór kondensatorów}
Kondensatory zostały dobrane zgodnie z zaleceniami producenta układu, jednocześnie biorąc pod uwagę spadek pojemności. Wszystkie użyte kondensatory to kondensatory MLCC.

\subsection{Filtr}
Na wyjściu układu umieszczony został filtr LP w topologii \Pi. Jego zadaniem jest usunięcie składowych wysoko-częstotliwościowych z napięcia wyjściowego, co przekłada się na zmniejszenie tętnień. W celu minimalizacji kosztów wykorzystana została taka sama cewka, jak w sekcji cyfrowej, wybrane zostały również wykorzystane wcześniej kondensatory \SI{10}{\micro\farad}. Pozwoliło to na uzyskanie częstotliwości granicznej: $$f_c = \frac{1}{2\pi \cdot \sqrt{L \cdot C}} = \frac{1}{2\pi \cdot \sqrt{\SI{3.3}{\micro\henry} \cdot \SI{10}{\micro\farad}}} = \SI{27.7}{\kilo\hertz}$$
W wyniku spadku pojemności kondensatorów rzeczywista częstotliwość graniczna wynosi około $\SI{35}{\kilo\hertz}$. Jest to zmiana mająca pomijalny wpływ na działanie układu, jednakże w celu zbliżenia filtru do początkowo planowanego umieszczone zostały po dwa kondensatory na wejściu i wyjściu filtru. Dodanie kolejnego typu kondensatora byłoby zbędnym zwiększaniem kosztów produkcji układu.

\subsection{Wyniki symulacji}
\begin{table}[H]
    \centering
    \caption{Porównanie wyników symulacji w zależności od obciążenia}
    \label{tab:efficiency_ripple}
    \begin{tabular}{ccccc}
        \toprule
        \textbf{Prąd obciążenia [mA]} & \textbf{Sprawność [\%]} & $\mathbf{\Delta{V_{IN}} \ [mV]}$ & $\mathbf{\Delta{V_{OUT}} \ [mV]}$ & $\mathbf{V_{AVG} \ [V]}$ \\
        \midrule
        0.0                           & 0                       & 0                                & 0.36                              & 12.27                    \\
        10.0                          & 96.46                   & 7.75                             & 5.58                              & 11.95                    \\
        20.0                          & 96.36                   & 8.36                             & 3.93                              & 11.95                    \\
        30.0                          & 95.75                   & 8.62                             & 4.95                              & 11.95                    \\
        40.0                          & 95.27                   & 9.17                             & 6.49                              & 11.95                    \\
        50.0                          & 94.61                   & 8.19                             & 4.03                              & 11.95                    \\
        60.0                          & 95.59                   & 7.14                             & 4.97                              & 11.95                    \\
        70.0                          & 96.04                   & 1.70                             & 2.40                              & 11.94                    \\
        80.0                          & 95.47                   & 1.95                             & 2.43                              & 11.94                    \\
        90.0                          & 95.75                   & 2.34                             & 2.24                              & 11.94                    \\
        100.0                         & 96.19                   & 2.23                             & 1.43                              & 11.94                    \\
        \bottomrule
    \end{tabular}
\end{table}

\subsection{Wnioski}
Wykorzystanie przetwornicy SEPIC pozwoliło uzyskać bardzo małe tętnienia napięcia wyjściowego. Udało się również uzyskać bardzo dużą sprawność układu.

% --------------------------------

\section{Sekcja analogowa III}
\subsection{Opis układu}
Zaprojektowana przetwornica podnosi i odwraca napięcie z \SI{10}{\V} do \SI{-12}{\V} i działa dla prądu maksymalnego \SI{100}{\milli\A}. Jej zadaniem jest zasilanie analogowej sekcji układu, która powinna mieć jak najmniejsze tętnienia. Układ wykorzystuje przetwornicę LT8362. Jest to układ analogiczny do poprzedniego, dzięki wykorzystaniu topologii SEPIC uzyskanie ujemnego napięcia wymaga minimalnej zmiany układu.

\begin{figure}[H]
    \centering
    \captionsetup{justification=centering}
    \begin{adjustbox}{width=1.2\textwidth,center}
        \subfloat[Schemat układu]{
            \includegraphics[width=\textwidth]{./figures/analog_3/schematic.png}
            \label{fig:subfig1}
        }
        \qquad
        \subfloat[Przebiegi czasowe prądu i napięcia na wyjściu układu \\ przy maksymalnym obciążeniu]{
            \begin{tikzpicture}
                \pgfplotsset{
                    scale only axis,
                    scaled x ticks=base 10:3,
                    xmin=0, xmax=0.01
                }

                \begin{axis}[
                        grid=both,
                        axis y line*=left,
                        ymin=-13, ymax=1,
                        xlabel={Czas $t$ [s]},
                        ylabel={Napięcie wyjściowe $V_{\text{OUT}}$ [V]},
                    ]
                    \addplot[green, thick] table {./figures/analog_3/vout/segment_1.csv};\label{vout}
                    \addplot[green, thick] table {./figures/analog_3/vout/segment_2.csv};
                    \label{plot:vout}
                \end{axis}

                \begin{axis}[
                        axis y line*=right,
                        axis x line=none,
                        ymin=-0.01,
                        ymax=0.13,
                        ytick pos=right,
                        yticklabel pos=right,
                        ytick = {-0.02, 0, 0.02, 0.04, 0.06, 0.08, 0.1, 0.12},
                        yticklabels = {-20, 0, 20, 40, 60, 80, 100, 120},
                        ylabel={Prąd wyjściowy $I_{\text{OUT}}$ [mA]},
                        legend style={at={(0.95,0.05)},anchor=south east},
                    ]
                    \addlegendimage{/pgfplots/refstyle=vout}
                    \addlegendentry{$V_{OUT}$}
                    \addplot[red, thick] table {./figures/analog_3/iout/segment_1.csv};\label{iout}
                    \addplot[red, thick] table {./figures/analog_3/iout/segment_2.csv};

                    \addlegendimage{/pgfplots/refstyle=iout}
                    \addlegendentry{$I_{OUT}$}
                    \label{plot:current}
                \end{axis}
            \end{tikzpicture}

            \label{fig:subfig2}
        }
    \end{adjustbox}
\end{figure}

\subsection{Wybór elementów}
Ten układ jest analogiczny do poprzedniego, dlatego wykorzystano te same elementy. Wykorzystanie topologii SEPIC pozwoliło na użycie tych samych elementów, co w przypadku przetwornicy nieodwracającej.
Jedyną rożnicą jest zamiana miejscami diody z cewką oraz przeprojektowanie dzielnika programującego napięcie wyjściowe. W tym przypadku został on zaprojektowany zgodnie ze wzorem $R_1 = R_2 \cdot \left(\frac{\lvert V_{OUT} \rvert}{\SI{0.8}{\V} - 1}\right)$

\subsection{Wyniki symulacji}
\begin{table}[H]
    \centering
    \caption{Porównanie wyników symulacji w zależności od obciążenia}
    \label{tab:efficiency_ripple}
    \begin{tabular}{ccccc}
        \toprule
        \textbf{Prąd obciążenia [mA]} & \textbf{Sprawność [\%]} & $\mathbf{\Delta{V_{IN}} \ [mV]}$ & $\mathbf{\Delta{V_{OUT}} \ [mV]}$ & $\mathbf{V_{AVG} \ [V]}$ \\
        \midrule
        0.0                           & 0.00                    & 0.00                             & 0.76                              & -12.582                  \\
        10.0                          & 100.35                  & 11.97                            & 15.55                             & -12.005                  \\
        20.0                          & 97.88                   & 14.97                            & 19.46                             & -12.005                  \\
        30.0                          & 97.09                   & 16.89                            & 19.17                             & -12.005                  \\
        40.0                          & 96.49                   & 17.29                            & 14.32                             & -12.005                  \\
        50.0                          & 95.90                   & 14.91                            & 12.71                             & -12.005                  \\
        60.0                          & 96.04                   & 11.57                            & 10.68                             & -12.005                  \\
        70.0                          & 95.95                   & 1.75                             & 2.78                              & -12.005                  \\
        80.0                          & 95.95                   & 1.91                             & 2.75                              & -12.004                  \\
        90.0                          & 95.83                   & 2.19                             & 2.33                              & -12.004                  \\
        100.0                         & 95.73                   & 2.35                             & 2.93                              & -12.004                  \\
        \bottomrule
    \end{tabular}
\end{table}

\subsection{Wnioski}
Wykorzystanie przetwornicy SEPIC znacząco ułatwiło projektowanie układu. Wykorzystanie tych samych elementów ma zaletę ze względu na mniejsze koszty produkcji. Udało się uzyskać bardzo małe tętnienia napięcia wyjściowego, co jest kluczowe w przypadku zasilania analogowej sekcji układu. Problemem przy tej symulacji jest liczenie sprawności, która jest niefizyczna, ponieważ moc wyjściowa potrafi być większa niż wejściowa - byłem w tej kwestii na konsultacjach. Późniejsze testy wykazały, że spowodowane jest to modyfikacją parametru Trtol w LTSpice, który odpowiada za dokładność obliczeń. Niestety ustawienie domyślnej wartości nie pozwalało na ukończenie symulacji w rozsądnym czasie - jedna symulacja, z jedenastu potrzebnych do wykonania, trwała około 3 godzin.

\section{Wnioski i podsumowanie}
Projektowanie układów zasilających wymaga wyboru między sprawnością, a stabilnością napięcia wyjściowego. Wykorzystanie stabilizatorów liniowych pozwala na uzyskanie bardzo małych tętnień, ale jest to robione kosztem sprawności, która w przybliżeniu wynosi $\eta = \frac{V_{OUT}}{V_{IN}}$. W układach, które są mniej wrażliwe na tętnienia, jak np. układy cyfrowe, warto zastosować przetwornice impulsowe, których sprawności sięgają rzędu \SI{90}{\percent}.\\
Podczas projektowania bardzo ważne okazuje się również uwzględnienie parametrów pasożytniczych, które mają bardzo duży wpływ na działanie układu. Przykładem może być tutaj przetwornica z sekcji cyfrowej, gdzie usunięcie kondensatora MLCC i pozostawienie jedynie kondensatora elektrolitycznego spowodowało podniesienie tętnień z kilkudziesięciu mV do kilkuset mV.\\
Wszystkie rezystory wykorzystane w przetwornicach pochodzą z szeregu E24. W przypadku gdy wartości w symulacjach nie mają swojego odpowiednika w szeregu E24, należy założyć łączenie rezystorów szeregowo w celu uzyskania odpowiednich wartości \textemdash \ łączenie rezystorów w jeden obiekt w LTSpice było elementem optymalizacji czasu symulacji poprzez zmniejszenie liczby symulowanych węzłów. Każdy rezystor (spoza szeregu, takie znajdują się np. w obwodach ustawiania częstotliwości) w symulacji da się utworzyć poprzez szeregowe połączenie dwóch rezystorów\\
\\Dokładna lista wszystkich elementów znajduje się w dołączonym do projektu pliku BOM.
\\

\noindent Całość projektu jest dostępna w repozystorium na GitHubie pod adresem:\\ \url{https://github.com/PiotrPokornowski/ELA2}

\end{document}
